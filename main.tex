%% LyX 2.3.6.2 created this file.  For more info, see http://www.lyx.org/.
%% Do not edit unless you really know what you are doing.
\documentclass[british,conference]{IEEEtran}
\usepackage[T1]{fontenc}
\usepackage[latin9]{inputenc}

\makeatletter
%%%%%%%%%%%%%%%%%%%%%%%%%%%%%% User specified LaTeX commands.
%IEEE SPS conference style file - a4 version
%\usepackage{spconfa4_noaddress}

%key parameters of document 
%\title{LyX environment works for this so leave it there}
%\name{Alastair H. Moore}
%\address{Imperial College London\\
% {\tt alastair.h.moore@imperial.ac.uk}}

%acronymns
\usepackage{relsize}
\usepackage[printonlyused,withpage]{acronym} % add 'smaller' option to make small caps
\newcommand{\acronymlist}[1]{
\begin{acronym}[MMMMMMM]\input{#1}\end{acronym}}
\newcommand{\acronymnolist}[1]{
\newcommand{\acro}{\acrodef}\newcommand{\acroindefinite}{\acrodefindefinite}\input{#1}}

%\usepackage{enumitem}
%\setitemize{noitemsep,topsep=0pt,parsep=0pt,partopsep=0pt}
%\setitemize{noitemsep}
%\setenumerate{noitemsep,leftmargin=*}


%symbols

\newcommand{\sigsrct}{s}
\newcommand{\idsrc}{n}
\newcommand{\numsrc}{N}
\newcommand{\sigmict}{x}
\newcommand{\idmic}{m}
\newcommand{\nummic}{M}
\newcommand{\airt}{h}
\newcommand{\signoiset}{\upsilon}

\newcommand{\idtime}{t}

\newcommand{\idmicref}{1}
\newcommand{\airrelt}{g}
\newcommand\signoiserelt{\tilde{\upsilon}}



\makeatother

\usepackage{babel}
\usepackage{listings}
\renewcommand{\lstlistingname}{Listing}

\begin{document}
\title{Acoustic scene awareness in realistic environments}
\author{Alastair H. Moore}

\maketitle
\acronymnolist{sapacronyms.txt}
\acro{ELOSPHERES}{ELOSPHERES}
\acro{ATF}{Acoustic Transfer Function}
\acro{RTF}{Relative Transfer Function}
\acro{HRTF}{Head-Related Transfer Function}

\section{Introduction\label{sec:intro}}

The purpose of this template is to give writers a starting point for
making a nice-looking document which is consistent across reports.
It was created in LyX, which give a word processor interface. But
the is a plain LaTeX version available too. If using LyX on mac you
wish to note the following
\begin{itemize}
\item The UI doesn't work well with dark mode
\item You can install using homebrew with 
\begin{lstlisting}
brew install --cask lyx
\end{lstlisting}
\item You can insert pure latex using ``evil red text'' by pressing command+L
\end{itemize}
The template has Mike's acronym macros already defined. Standard acronyms
are in sapacronyms.txt and additional acronyms can be defined by editing
that file, or simply adding them to the list at the top of this document.
So using some ERT we can define the acronym for \ac{RTF} on its first
use, after which it will appear as \ac{RTF}.

Maths is entered using the equation editor which allows you type latex
maths commands which it will typeset on-the-fly.

Non-printing notes can be added using the ``Insert LyX note'' button,
which looks like a post-it note.

Citations such as \cite{Zohourian2015,Knapp1976} are nicely handled.
Bib file sources are defined by clicking on the 'BibTeX Generated
Bibliography' button at the bottom of this document and add the paths
to local sources. 

\section{Some maths}

The microphone signals, $\sigmict_{\idmic}(\idtime)$, are related
to the source signals, $\sigsrct_{\idmic}(\idtime)$, according to
\begin{eqnarray*}
\sigmict_{\idmic}(\idtime) & = & \sum_{\idsrc=1}^{\numsrc}\left(\airt_{\idmic,\idsrc}(\idtime)\ast\sigsrct_{\idsrc}(\idtime)\right)+\signoiset_{\idmic}(\idtime)
\end{eqnarray*}

where...

\section{Challenges}

After \cite{Zohourian2015}, we will assume the availability of estimated
head movement data from an \ac{IMU}. This does not remove the problems
associated with non-stationarity. Rather it provides an additional
imperfect source of information which can inform our processing in
order to account for it. We believe that the incorporation of an \ac{IMU}
into hearing aids is a realistic assumption since such devices are
mass produced for modern smart phones. We plan to postpone tackling
point 4 until later in the project, noting that one possible line
of enquiry would be to follow an iterative approach as proposed in
\cite{Romigh2012}. 

The remainder of this report considers the difficulties in estimating
multiple dynamic source directions relative to a moving listener in
reverberant environments using typical approaches to \ac{DOA} estimation
before proposing a research direction for the following 6 months.

\section{\acs{DOA} estimation from estimated \acs{RTF}}

To mitigate the effect of reverberation, it has been proposed to
estimate \ac{TDOA} from estimates of either the \acp{ATF} \cite{Kowalczyk2013}
or the \ac{RTF}, with the latter being a more robust approach. The
\ac{TOA} of the first peak in the \ac{RTF} is then the \ac{TDOA}
with respect to the reference channel. This approach is promising
and \ac{RTF} estimation is an active area of research \cite{Cohen2004a,Markovich2009,Taseska2015}.
However, current techniques assume that sources and receivers are
fixed or move only very slowly and so would not adapt quickly enough
to the rapid changes expected due to head movement.  Estimation of
the \ac{RTF} when multiple sources are simultaneously active is also
an unsolved problem.

\section{\acs{DOA} estimation using \acs{SRP}}

The idea of pointing a beamformer in all possible source directions
and selecting those that lead to the largest output as the \acp{DOA}
is old but remains popular due to its inherent robustness and its
ability to identify multiple \acp{DOA}. However, the spatial selectivity
is generally lower than other methods and depends on the choice of
beamformer. The \ac{MVDR} beamformer is the theoretically optimal
choice, since it adjusts the beampattern to reject background noise
based on the noise covariance matrix. With the steering vector set
to the anechoic \ac{ATF} for a planewave in the look direction, the
beamformer gain for the direct path is unity. However, in reverberation,
coherent reflections can lead to cancellation of the desired source
due to negative lobes directed towards the reflections. To avoid such
cancelation requires the steering vector to be the \ac{ATF} of the
desired source, which is generally unknown, or the \ac{RTF}, which
may be estimated, as discussed above.

\section{Next steps}

\bibliographystyle{IEEEtran}
\bibliography{sapstrings,sapref,local}

\end{document}
