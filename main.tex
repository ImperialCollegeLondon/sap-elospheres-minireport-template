%% LyX 2.3.6.2 created this file.  For more info, see http://www.lyx.org/.
%% Do not edit unless you really know what you are doing.
\documentclass[british,conference]{IEEEtran}
\usepackage[T1]{fontenc}
\usepackage[latin9]{inputenc}
\usepackage{babel}
\usepackage[unicode=true]
 {hyperref}

\makeatletter
%%%%%%%%%%%%%%%%%%%%%%%%%%%%%% User specified LaTeX commands.
%IEEE SPS conference style file - a4 version
%\usepackage{spconfa4_noaddress}

%key parameters of document 
%\title{LyX environment works for this so leave it there}
%\name{Alastair H. Moore}
%\address{Imperial College London\\
% {\tt alastair.h.moore@imperial.ac.uk}}

%acronymns
\usepackage{relsize}
\usepackage[printonlyused,withpage]{acronym} % add 'smaller' option to make small caps
\newcommand{\acronymlist}[1]{
\begin{acronym}[MMMMMMM]\input{#1}\end{acronym}}
\newcommand{\acronymnolist}[1]{
\newcommand{\acro}{\acrodef}\newcommand{\acroindefinite}{\acrodefindefinite}\input{#1}}

%\usepackage{enumitem}
%\setitemize{noitemsep,topsep=0pt,parsep=0pt,partopsep=0pt}
%\setitemize{noitemsep}
%\setenumerate{noitemsep,leftmargin=*}


%symbols

\newcommand{\sigsrct}{s}
\newcommand{\idsrc}{n}
\newcommand{\numsrc}{N}
\newcommand{\sigmict}{x}
\newcommand{\idmic}{m}
\newcommand{\nummic}{M}
\newcommand{\airt}{h}
\newcommand{\signoiset}{\upsilon}

\newcommand{\idtime}{t}

\newcommand{\idmicref}{1}
\newcommand{\airrelt}{g}
\newcommand\signoiserelt{\tilde{\upsilon}}



\makeatother

\usepackage{listings}
\lstset{basicstyle={\ttfamily}}
\renewcommand{\lstlistingname}{Listing}

\begin{document}
\title{A template for minireports}
\author{Alastair H. Moore}

\maketitle
\acronymnolist{sapacronyms.txt}
\acro{ELOSPHERES}{ELOSPHERES}
\acro{ATF}{Acoustic Transfer Function}
\acro{RTF}{Relative Transfer Function}
\acro{HRTF}{Head-Related Transfer Function}

\section{Introduction\label{sec:intro}}

The purpose of this template is to give writers a starting point for
making a nice-looking document which is consistent across reports.
It was created in LyX, which gives a word processor interface. But
the is a plain LaTeX version available too which was generated from
the lyx file. If using LyX on mac you may wish to note the following
\begin{itemize}
\item You can install using homebrew with 
\begin{lstlisting}
brew install --cask lyx
\end{lstlisting}
 It will appear in your Applications folder. You may need to right
click->Open it the first time you run it.
\item The UI doesn't work well with dark mode - in LyX go to LyX -> Preferences...
-> Look \& Feel -> Colors and uncheck ``use system colors''. Also
turn off dark mode Apple menu -> System Preferences -> General ->
Appearance -> Light.
\end{itemize}
The UI has buttons for most common formatting operations. You can
insert pure latex using ``evil red text'' by pressing cmd+L.

The template has Mike's acronym macros already defined. Standard acronyms
are in sapacronyms.txt and additional acronyms can be defined by editing
that file, or simply adding them to the list at the top of this document.
So using some ERT we can define the acronym for \ac{RTF} on its first
use, after which it will appear as \ac{RTF}.

Maths is entered using the equation editor which allows you type latex
maths commands which it will typeset on-the-fly.

Non-printing notes can be added using the ``Insert LyX note'' button,
which looks like a post-it note.

Citations such as \cite{Zohourian2015,Knapp1976} are nicely handled.
Bib file sources are defined by clicking on the 'BibTeX Generated
Bibliography' button at the bottom of this document and add the paths
to local sources. 

\section{Some maths}

The microphone signals, $\sigmict_{\idmic}(\idtime)$, are related
to the source signals, $\sigsrct_{\idmic}(\idtime)$, according to
\begin{eqnarray*}
\sigmict_{\idmic}(\idtime) & = & \sum_{\idsrc=1}^{\numsrc}\left(\airt_{\idmic,\idsrc}(\idtime)\ast\sigsrct_{\idsrc}(\idtime)\right)+\signoiset_{\idmic}(\idtime)
\end{eqnarray*}

where...

\section{Next steps}
\begin{itemize}
\item Edit your document
\item To View the output do cmd+r or to export as pdf do File->Export->PDF
(pdflatex)
\item If you want to automatically create the pdf whenever you view then
have a look at \href{https://tex.stackexchange.com/questions/89561/shortcut-for-export-pdf-on-lyx}{this stack exchange answer}.
\end{itemize}
\bibliographystyle{IEEEtran}
\bibliography{sapstrings,sapref,local}

\end{document}
